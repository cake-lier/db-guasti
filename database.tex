\documentclass[a4paper, 12pt]{report}

\usepackage[italian]{babel}
\usepackage{graphicx}
\usepackage{float}
\usepackage{tabularx}
\usepackage{ltablex}
\usepackage[font=small,format=plain,labelfont=bf,up,textfont=normal,up,justification=justified,singlelinecheck=false,skip=0.01\linewidth]{caption}
\usepackage{enumitem}
\renewcommand{\familydefault}{\sfdefault}

\title{``Progetto di una base di dati per la gestione del ciclo di vita del difetto nella produzione di un elettrodomestico"}
\author{Castellucci Matteo\\Matricola 0000825436\\Anno accademico 2018-2019}
\date{\today}

\begin{document}

\maketitle

\tableofcontents

\chapter{Introduzione}
In questo progetto si vuole realizzare una base di dati a supporto della gestione dei difetti nella realizzazione di un elettrodomestico,
chiudendo la "catena dell'informazione" tra cliente ed azienda nella fase terminale del ciclo di vita del prodotto. Lo scopo è fornire
all'azienda tutte le informazioni necessarie sugli elettrodomestici dopo essere arrivati nelle mani del cliente finale, così da poter
ottenere informazioni sul campo e capire quali sono gli obiettivi portati a termine e quali no. In particolar modo, l'attenzione è posta sull'elemento
principe per quanto riguarda l'analisi del comportamento di un bene di consumo: il difetto, in tutti i suoi aspetti. Sarà permesso a tutti gli
attori in gioco - tecnici, telefonisti, analisti dei dati e progettisti - di poter avere una visione a tutto tondo dei problemi che riguardano
il prodotto: le tempistiche di manifestazione, di risoluzione, il tipo di guasto ed altro ancora. Questo permetterà all'azienda di poter
avere più controllo sullo sviluppo e la messa in produzione di futuri progetti avendo a disposizione più informazioni strategiche.

\chapter{Analisi dei requisiti}

\section{Intervista}
Si vuole realizzare una base di dati che permetta di gestire tutte le informazioni che riguardano un guasto di un elettrodomestico.
L'azienda riceve telefonate da un cliente che vengono smistate ad un adeguato centro assistenza dove ciascuna verrà raccolta da un operatore.
Ogni centro assistenza ha sede in una città e possiede una determinata area di competenza, per cui tutte le chiamate che verranno effettuate
all'interno di quest'ultima verranno redirette al centro assistenza associato. L'identificazione dei centri assistenza viene fatta mediante dei
codici che hanno però valenza solamente nazionale.\newline
L'operatore, identificato mediante un codice univoco tra i dipendenti, di cui l'azienda conosce dati identificativi come nome, cognome, data
e luogo di nascita, codice fiscale, luogo di residenza, nonché lo stipendio che gli elargisce, così come per ogni altro dipendente, chiede al 
cliente di identificarsi. Vengono richiesti dall'operatore 
il nome, il cognome e il recapito a cui fare riferimento nel momento nel quale un solo tecnico si recherà in loco per riparare il suo elettrodomestico. 
Eventualmente, l'operatore chiede anche l'indirizzo \textit{email}, qualora il cliente lo possieda. Inoltre, viene registrato il numero di telefono 
chiamante a cui verrà fatto riferimento anche in futuro come specifico per quel dato cliente.\newline
In seguito, l'operatore chiede quale prodotto o quali prodotti del cliente hanno subito un guasto e saranno l'oggetto della corrente richiesta di 
assistenza. L'operatore per ogni elettrodomestico ne chiede la categoria, il modello, specificato con un codice, e si fa leggere "PNC" (\textit{Product 
Number Code}) e "SNC" (\textit{Serial Number Code}), codici che lo identificano univocamente trattenendo informazioni sulle caratteristiche tecniche e
la data di produzione. L'operatore richiede inoltre la data di acquisto dell'elettrodomestico e la data di installazione dello stesso e le registra qualora 
il cliente le abbia a disposizione o se ne ricordi. In assenza di data di acquisto sarà compito del tecnico, una volta recatosi a casa del cliente, 
provare a recuperare queste informazioni e in caso la garanzia del prodotto sia scaduta o non sia trovata, far pagare il cliente. Il cliente potrebbe avere 
aderito opzionalmente ad un programma di estensione della garanzia, identificato da un codice, informazione che deve fornire, se non fase in chiamata, 
almeno al momento della visita. Da ultimo, l'operatore chiede al cliente di farsi descrivere il guasto o i guasti subiti e registra la data corrente come data di apertura 
della pratica, ponendo lo stato dell'intervento come aperto e concorda con il cliente la data di visita del tecnico a casa per la riparazione.\newline
Un difetto è univocamente determinato da due codici, il "\textit{Component Code}", che identifica quale parte di un elettrodomestico viene affetto, e
il codice di tipo, che indica qual è nello specifico il tipo di difetto che il componente può subire. Ad ogni difetto sono associati i ricambi che sono 
necessari per poterlo sistemare, che potrebbero anche essere più di uno o nessuno, identificati da un codice, e il costo di ciascuno di essi, 
sia in termini di costo del ricambio stesso, che del costo di manodopera per l'installazione in fase di costruzione.\newline
Un tecnico, che afferisce ad un centro assistenza come l'operatore, si preoccuperà, una volta raccolte le informazioni necessarie dalla chiamata,
di recarsi a casa del cliente e registrare a sua volta una descrizione personale del guasto o dei guasti del cliente, dopodiché possibilmente li riparerà
e si preoccuperà di chiudere la richiesta di assistenza pendente. Specificherà anche il tempo che ha impiegato nel risolvere il problema del cliente
espresso in quarti d'ora, ma non i dati sull'eventuale pagamento effettuato dal cliente stesso nei suoi confronti, essendo irrilevanti per l'analisi dei dati successiva.
Un tecnico può lavorare direttamente per l'azienda o essere assunto tramite un'impresa terza per l'azienda stessa. \newline
Un progettista di elettrodomestici dovrà poi periodicamente recuperare le informazioni riguardanti i guasti che si sono verificati sui prodotti che
gli competono, indicati dalla o dalle categorie di prodotto a cui è assegnato, e stilare classifiche che possano aiutare l'azienda a migliorare la produzione.

\section{Analisi ed eliminazione delle ambiguità}

Il termine "elettrodomestico" verrà da qui in avanti sostituito dal termine "prodotto", in quanto dotato di un più adeguato carattere di astrattezza
e generalità, benché il caso in esame coinvolga un'azienda che produce elettrodomestici. I termini come "chiamata", "richiesta di
assistenza", "intervento" e loro sinonimi vengono tutti raccolti sotto il termine "intervento" poiché di fatto descrivono le varie fasi dello
stesso, dalla richiesta iniziale del cliente, alla visita del tecnico dell'assistenza, fino alla riparazione del prodotto indicato. Si intende
che ogni intervento sarà raccolto in un'unica pratica e perciò anche quest'ultimo termine indica lo stesso concetto.

\begin{tabularx}{\linewidth}{X|X}
	\hline
	\textbf{Termine presente nel testo} & \textbf{Sinonimo utilizzato}\\
	\hline
	\hline
	Elettrodomestico & Prodotto\\
	\hline
	Pratica & Intervento\\
	\hline
	Telefonata & Intervento\\
	\hline
	Richiesta di assistenza & Intervento\\
	\hline
	Chiamata & Intervento\\
	\hline
	Riparazione & Intervento\\
	\hline
	\caption{Associazioni termine-sinonimo}
\end{tabularx}

\section{Definizione delle specifiche ed estrapolazione dei concetti}
Si vuole realizzare una base di dati che permetta di gestire tutte le informazioni che riguardano un \textbf{\underline{guasto}} di un \textbf{\underline{prodotto}}.
Un prodotto è univocamente identificato da una combinazione di codici detti "PNC" (\textit{Product Number Code}) e "SNC" (\textit{Serial Number Code}) ed è caratterizzato
da una \textbf{\underline{categoria}} e un \textbf{\underline{modello}} sotto forma di codice. Un prodotto possiede inoltre una data di acquisto e una di installazione, insieme ad 
un'estensione di garanzia identificata da un codice, informazioni che però possono essere disponibili o meno, a seconda che il \textbf{\underline{cliente}}
che lo possiede ne abbia tenuto traccia o meno.\newline
Al verificarsi di un guasto, il cliente si preoccupa di chiamare l'assistenza clienti dell'azienda produttrice per richiedere un
\textbf{\underline{intervento}}. Ogni cliente verrà identificato dal suo numero di telefono e verranno registrati il suo nome, il suo cognome,
il recapito presso il quale inviare il \textbf{\underline{tecnico}} per riparare il prodotto ed opzionalmente il suo indirizzo \textit{email},
solamente nel caso il cliente lo possedesse.\newline
Il cliente parlerà con un \textbf{\underline{operatore}}, che chiederà al cliente tutte le informazioni relative al prodotto o ai prodotti
che hanno subito un guasto insieme alle descrizioni dei guasti stessi che vengono anch'esse registrate. Viene inoltre concordata una data di
visita del tecnico affinché possa riparare il prodotto. Infine, vengono registrati la data della chiamata del cliente e viene
aggiornato lo stato dell'intervento come aperto. Quando il tecnico visiterà il cliente, uno solo per intervento, registrerà una sua descrizione
del guasto per ognuno di essi e dopo averli riparati segnerà lo stato dell'intervento come chiuso. Indicherà anche il tempo speso per farlo in quarti d'ora.\newline
Un guasto è associato ad uno specifico \textbf{\underline{difetto}}, il quale è univocamente determinato da due codici: il "\textit{Component Code}"
che identifica il componente del prodotto coinvolto e un codice che identifica il tipo del difetto su quello specifico componente. Nella
riparazione di un difetto possono essere utilizzati uno o più \textbf{\underline{ricambi}} e nel caso il tipo di difetto li richiedesse, il loro uso non è opzionale.
Ognuno dei ricambi è identificato da un codice e vi è associato sia un costo di acquisto per il ricambio stesso che di manodopera per l'installazione.\newline
L'azienda possiede per i suoi dipendenti informazioni come il codice fiscale, il nome, il cognome, la data e il luogo di nascita, luogo di
residenza e lo stipendio percepito e li identifica mediante un codice univoco tra tutti i dipendenti. I dipendenti presi in considerazione sono gli operatori, i tecnici e i
\textbf{\underline{progettisti}}. Gli operatori e i tecnici afferiscono ad uno specifico \textbf{\underline{centro assistenza}} dell'azienda,
mentre i progettisti sono interessati ad analizzare i dati dei guasti di una o più categorie di prodotto stilando delle classifiche. Un tecnico
inoltre può essere dipendente interno dell'azienda oppure no.\newline
Un centro assistenza ha sede in una città e possiede una determinata area di competenza, nella quale tutte le telefonate che vengono fatte
per l'assistenza vengono inoltrate a quello specifico centro. Esso è identificato da un codice univoco, ma solo per la nazione in cui si trova.\newline

\begin{tabularx}{\linewidth}{>{\hsize=0.375\hsize}X|X|>{\hsize=0.475\hsize}X}
	\hline
	\textbf{Termine} & \textbf{Descrizione} & \textbf{Collegamenti}\\
	\hline
	\hline
	Prodotto & Un elettrodomestico di un determinata categoria che l'azienda ha rilasciato sul mercato ed è stato comprato da un cliente, per poi subire un guasto &
	Cliente,\newline Categoria,\newline Guasto\\
	\hline
	Cliente & Una persona che ha acquistato un prodotto presso l'azienda e ha richiesto un intervento per quello specifico prodotto & Prodotto,\newline 
	Intervento\\
	\hline
	Guasto & Un problema di varia natura che si è verificato in uno specifico prodotto, causato da un difetto nel prodotto stesso, per cui viene
	richiesto un intervento. Sarà analizzato da un progettista & Prodotto,\newline Difetto,\newline Intervento,\newline Progettista\\
	\hline
	Intervento & Una richiesta fatta da un cliente per risolvere uno o più guasti subiti ai propri elettrodomestici, sarà registrato da un operatore e risolto da un tecnico
	& Guasto,\newline Operatore,\newline Tecnico,\newline Cliente\\
	\hline
	Difetto & Una classe di guasti accomunati dal componente in cui il guasto si verifica e la natura del difetto stesso, possono essere utilizzati
	dei ricambi per poterlo riparare & Guasto,\newline Ricambio\\
	\hline
	Ricambio & Un componente che può essere utilizzato nella riparazione di un difetto & Difetto\\
	\hline
	Operatore & Un dipendente dell'azienda che si preoccupa di aprire richieste di intervento, appartiene ad un determinato centro assistenza
	& Intervento,\newline Centro Assistenza\\
	\hline
	Tecnico & Un dipendente dell'azienda che si preoccupa di attendere ad interventi di manutenzione e chiudere le richieste pendenti, appartiene
	ad un centro assistenza & Intervento,\newline Centro Assistenza\\
	\hline
	Progettista & Un dipendente dell'azienda che si preoccupa di analizzare i dati sui guasti accaduti, ma solamente su alcune categorie
	di prodotto & Guasto,\newline Categoria\newline Prodotto\\
	\hline
	Centro Assistenza & Parte dell'azienda volta all'impiego di operatori e tecnici di assistenza & Operatore,\newline Tecnico\\
	\hline
	Categoria & Un'insieme di prodotti che rappresentano lo stesso tipo di elettrodomestico (ad esempio: "Forno", "Lavatrice", ecc.) 
	e viene usata dai progettisti per suddividersi i guasti da analizzare & Prodotto,\newline Progettista\\
	\hline
	\caption{Glossario dei termini}
\end{tabularx}

\chapter{Progettazione concettuale}

\section{Schema scheletro}

\begin{figure}[H]
	\centering
	\includegraphics[width=\linewidth]{images/skeleton.png}
	\caption{Schema scheletro dei concetti più rilevanti estrapolati dalle specifiche}
\end{figure}

\newpage

\section{Raffinamenti proposti}

\subsection{Gerarchia di persone}

Nelle specifiche sono indicati quattro concetti - "cliente", "operatore", "tecnico" e "progettista" - che si prestano a diventare entità, ma
che ruotano tutte attorno alla nozione di "persona". Tutte queste infatti sono persone e in quanto tali condividono almeno le informazioni come il nome
ed il cognome; è perciò significativo metterle assieme in una gerarchia che discende da "persona". Inoltre, tutti e tre i concetti meno che "cliente"
sono di fatto dipendenti dell'azienda e in quanto tali le informazioni che l'azienda trattiene su di essi sono le stesse, pur intrattenendo con gli
altri concetti rapporti diversi a seconda della mansione che svolgono. Anche per questi tre concetti è opportuno raccoglierli attraverso una seconda
gerarchia che origini da dipendente, che a sua volta è una persona. I concetti di "operatore" e "tecnico" saranno poi legati a "centro assistenza"
così come individuato nello schema scheletro. Al "progettista" invece verranno associate mediante una relazione adeguata la o le categorie di prodotto 
di cui si occupa per quanto riguarda l'analisi dei guasti.

\begin{figure}[H]
	\centering
	\includegraphics[width=\linewidth]{images/persone.png}
	\caption{Schema raffinato dei concetti riguardanti "persone"}
\end{figure}

\newpage

\subsection{Necessità di un entità separata per i guasti}

Dai concetti di "persone" è poi immediato spostarsi sui quelli di "intervento" e "guasto" che li coinvolgono più direttamente. Benché il testo
specifichi che siano i prodotti, a causa dei loro guasti, ad essere in numero di uno o più presenti in un intervento, è più adatto frapporre il
concetto di "guasto" tra essi e quello di "intervento". Se non lo facessimo, questo provocherebbe problemi con la nozione di "difetto". Potendo
un prodotto essere afflitto anche da più guasti, infatti, potrebbero essere più di uno i difetti legati ad un prodotto, anche ripetuti nel tempo,
ma una schematizzazione del genere impedirebbe proprio la loro ripetizione. Potrebbe ad esempio verificarsi un guasto diverso in un momento successivo,
che verrà incluso poi in un intervento successivo, ma la cui origine è sempre lo stesso difetto di progettazione.\newline I dati che sono di competenza del tecnico
inserire, nonché i dati che lo riguardano, sono tutti modellati mediante attributi opzionali o relazioni con cardinalità minima zero, poiché si suppone che l'apertura della pratica coincida con 
l'inserimento di un nuovo \textit{record} nel \textit{database}, momento nel quale questi dati sono ancora sconosciuti.
La data di apertura e di chiusura di un intervento vengono realizzate mediante un attributo composto obbligatorio perché vogliamo esplicitamente specificare che non vogliamo
trattenere soltanto la data in quanto tale, ma anche l'ora, così da poter essere sicuri di distinguere anche i casi in cui abbiamo clienti che chiamano per più interventi nello
stesso giorno.

\begin{figure}[H]
	\centering
	\includegraphics[width=\linewidth]{images/interventi.png}
	\caption{Schema raffinato dei concetti "intervento", "guasto", "difetto" e ricambio"}
\end{figure}

\subsection{Modellazione del concetto di "prodotto"}

Da ultimo, espandiamo la nozione di "prodotto". Non è una componente dello schema particolarmente significativa, anche se presenta alcune particolarità.
Il fatto che il cliente possa non avere a disposizione le informazioni riguardanti la data di acquisto, installazione o addirittura la garanzia sul
prodotto, richiede una modellazione specifica. In particolar modo, queste informazioni sarà necessario modellarle come attributi singoli opzionali.

\begin{figure}[H]
	\centering
	\includegraphics[width=\linewidth]{images/prodotti.png}
	\caption{Schema raffinato del concetto "prodotto"}
\end{figure}

\section{Schema finale}

\begin{figure}[H]
	\centering
	\includegraphics[width=\linewidth]{images/conceptual.png}
	\caption{Schema concettuale finale}
\end{figure}

I vincoli che questo schema non può modellare sono chiaramente quelli inerenti allo stato delle entità nel tempo, non è infatti lo schema adatto per poter
specificare chi e come dovrà modificare lo stato dell'intervento una volta creatone uno nuovo. Inoltre, non è possibile vincolare il fatto che la data di
apertura dell'intervento dovrà essere chiaramente successiva a quella di chiusura dello stesso ed entrambe dovranno essere successive a quelle di acquisto e
installazione dei prodotti coinvolti nell'intervento.

\chapter{Progettazione logica}

\section{Volume dei dati}

\begin{tabularx}{\linewidth}{>{\hsize=0.375\hsize}X|X|>{\hsize=0.475\hsize}X}
	\hline
	\textbf{Nome del concetto} & \textbf{Tipo di costrutto} & \textbf{Stima del volume}\\
	\hline
	\hline
	Persona & E & 155.250\\
	\hline
	Dipendente & E & 450\\
	\hline
	Cliente & E & 154.800\\
	\hline
	Operatore & E & 80\\
	\hline
	Tecnico & E & 240\\
	\hline
	Progettista & E & 130\\
	\hline
	Centro Assistenza & E & 80\\
	\hline
	Afferenza\_Tec & R & 240\\
	\hline
	Afferenza\_Op & R & 80\\
	\hline
	Intervento & E & 158.400\\
	\hline
	Richiesta & R & 158.400\\
	\hline
	Apertura & R & 158.400\\
	\hline
	Assistenza & R & 158.400\\
	\hline
	Motivazione & R & 158.640\\
	\hline
	Guasto & E & 158.640\\
	\hline
	Analisi & R & 2.577.900\\
	\hline
	Difetto & E & 50\\
	\hline
	Causa & R & 158.640\\
	\hline
	Ricambio & E & 75\\
	\hline
	Riparazione & R & 75\\
	\hline
	Prodotto & E & 156.000\\
	\hline
	Effetto & R & 158.640\\
	\hline
	Categoria & E & 12\\
	\hline
	Appartenenza & R & 156.000\\
	\hline
	Interesse & R & 195\\
	\hline
	Proprietà & R & 156.000\\
	\hline
	\caption{Tabella dei volumi dei dati}
\end{tabularx}

Questa tabella prende in considerazione i volumi dei dati stimati nell'arco di un anno, immaginando che ogni mese solamente i dati inerenti ai guasti
degli undici mesi precedenti vengano mantenuti, mentre i dati vecchi di un anno o più vengano mantenuti ma solo in modo riassuntivo. In questo modo
possiamo assumere che i volumi dei dati indicati restino all'incirca costanti nel tempo.

\section{Descrizione e frequenza delle principali operazioni}

Le principali operazioni che possono essere fatte sulla base di dati possono essere divise in tre categorie, a seconda del dipendente
dell'azienda che interessano. Se prendiamo in considerazione un operatore, le operazioni più frequenti che deve eseguire sono:

\begin{itemize}
	\item[\textbf{O1} -] Inserimento di un nuovo intervento
		\subitem Un operatore dovrà poter inserire un nuovo intervento 
	\item[\textbf{O2} -] Cancellazione di un intervento precedentemente aperto dall'operatore
		\subitem Un operatore dovrà poter cancellare un intervento aperto in precedenza da lui stesso perché accortosi di un errore commesso nell'inserimento
		oppure perché il cliente non desidera più ricevere alcun tipo di assistenza
\end{itemize}

Se consideriamo invece un tecnico, le operazioni più frequenti che riguardano l'esecuzione delle sue mansioni sono:

\begin{itemize}
	\item[\textbf{T1 -}] Vista di tutti gli interventi correntemente aperti, ordinati per urgenza
		\subitem Prima di iniziare a lavorare, un tecnico dovrà capire quali sono gli interventi sui quali potrà dedicarsi e per poter scegliere, deve poter avere a
		disposizione tutti quanti quelli aperti. In particolar modo, sarebbe più utile organizzarli per urgenza, qui intesa come tempo trascorso dalla data della chiamata
	\item[\textbf{T2 -}] Chiusura di un intervento
		\subitem Una volta recatosi a casa di un cliente, il tecnico avrà avuto la possibilità di aggiungere al sistema le informazioni sul prodotto che il cliente
		non era riuscito a fornire. Inoltre, dopo aver eseguito la riparazione, sarà necessario che aggiunga anche le informazioni sull'intervento di sua sola competenza
\end{itemize}

Da ultimo, ci sono i progettisti che necessiteranno di stilare rapporti sotto forma di classifiche sulle principali caratteristiche dei guasti. Per poter migliorare le prestazioni,
non applichiamo il filtro sui dati a ciascuna operazione così da restituire solo i guasti di interesse ad un singolo progettista, ma eseguiamo le operazioni sulla totalità
dei dati. Ci preoccuperemo di filtrare i dati solamente una volta che verranno richiesti dallo specifico progettista, mostrandogli una vista creata specificatamente per lui, così
da poter effettuare le operazioni in modalità "\textit{batch}", fuori orario rispetto a quando i dati prodotti verranno utilizzati, ed averli sempre pronti senza dover attendere.
Le principali operazioni che permettono di estrarre dati di interesse ai progettisti sono:

\begin{itemize}
	\item[\textbf{P1 -}] Visualizzare tutte le informazioni relative ai guasti accaduti nel mese corrente
		\subitem Ogni mese è richiesto fare un \textit{report} sulla produzione degli elettrodomestici ed è perciò necessario avere a disposizione tutti i dati inerenti a
		tutti i guasti che sono accaduti in questo periodo
	\item[\textbf{P2 -}] Visualizzare tutti i guasti che presentano una certa parola chiave nella descrizione scritta dal tecnico
		\subitem Per poter analizzare un po' per volta i guasti, è utile poter filtrare solamente quelli che presentano un certo tipo di problema specifico così come indicato
		dalle parole stesse del tecnico che l'ha riparato, senza però la precisione data dalla ricerca per difetto
	\item[\textbf{P3 -}] I primi cinque \textit{PNC} che hanno subito più guasti nel mese corrente
		\subitem Ogni modello coinvolge più codici di prodotto e per un'analisi di in maggior dettaglio è necessario poter avere a disposizione una classifica
		di guasti ordinata per \textit{PNC}, da cui estrarre i più colpiti
	\item[\textbf{P4 -}] I primi cinque \textit{Component Code} che sono stati coinvolti maggiormente nei guasti di questo mese
		\subitem Alcuni componenti si guastano più di altri indipendentemente dal prodotto sul quale sono collocati. Si vuole ottenere i primi cinque maggiormente presenti nei guasti di questo 
		mese
	\item[\textbf{P5 -}] I primi cinque ricambi più utilizzati per riparare i guasti di questo mese
		\subitem Uno stesso ricambio può essere utilizzato in riparazioni di diversi guasti. Quelli più notevoli saranno quelli il cui costo è quello che andrà ad incidere maggiormente su quello 
		totale dei ricambi acquistati per le riparazioni, quindi quelli utilizzati in maggior numero. Si rende perciò necessario stilare una classifica dei ricambi più utilizzati nei guasti del 
		mese corrente
	\item[\textbf{P6 -}] I primi cinque interventi più costosi a partire dai ricambi nel mese corrente
		\subitem Il costo di un intervento è principalmente determinato dal costo dei ricambi utilizzati nella sua riparazione, è perciò interessante
		determinare per l'analisi dei costi i cinque interventi più costosi del mese in termini di ricambi impiegati
	\item[\textbf{P7 -}] Le prime cinque zone per numero di guasti avvenuti in questo mese
		\subitem Si vuole anche dare una "localizzazione spaziale" ai guasti, cioè individuare quali zone, indicate tramite le aree di competenza dei centri assistenza,
		hanno un tasso di difettosità più alto. Per poterlo fare, è bene poter classificare le varie aree sulla base del numero di guasti riparati dai centri assistenza
		che coprono quelle zone nel mese corrente.
	\item[\textbf{P8 -}] I \textit{Time To Failure} rispetto alla data di acquisto di ciascun prodotto coinvolto nei guasti di questo mese
		\subitem Un altro indicatore spesso utilizzato nell'elaborazione dei guasti è il cosiddetto "\textit{Time To Failure}", cioè il tempo impiegato da un particolare elettrodomestico
		per subire un guasto. In questo caso è calcolato a partire dal momento in cui l'elettrodomestico è stato acquistato.
	\item[\textbf{P9 -}] I \textit{Time To Failure} rispetto alla data di installazione di ciascun prodotto coinvolto nei guasti di questo mese
		\subitem Come l'operazione precedente, solamente che il \textit{Time To Failure} è calcolato a partire dalla data di installazione del prodotto.
	\item[\textbf{P10 -}] Tempo medio di riparazione di un guasto per tipo di difetto
		\subitem Una distinzione per costo tra i guasti può essere fatta anche nei termini del tempo medio di riparazione dei guasti associati ad un particolare tipo di difetto,
		andando ad indicare su quali parti di ciascun prodotto è necessario focalizzarsi per ridurre i costi di manutenzione dei prodotti stessi
\end{itemize}

\newpage

Alle tre categorie di operazioni precedenti possiamo aggiungerne una quarta, la quale contiene operazioni che non vanno eseguite da una specifica categoria di dipendenti,
ma riguardano informazioni utili all'azienda in generale. Le operazioni contenute sono:

\begin{itemize}
	\item[\textbf{V1} -] Conteggio di tutti gli interventi aperti nel mese da ciascun operatore
		\subitem Per valutare l'efficienza e la preparazione degli operatori dei centri assistenza deve essere
		possibile visualizzare il conteggio degli interventi aperti nel mese corrente da ciascun operatore
	\item[\textbf{V2 -}] Conteggio di tutti gli interventi chiusi nel mese corrente da ciascun tecnico
		\subitem Per valutare l'efficienza e la preparazione dei tecnici dei centri assistenza deve essere
		possibile visualizzare il conteggio degli interventi chiusi nel mese corrente da ciascun tecnico
	\item[\textbf{V3 -}] Tempo medio di riparazione di un guasto per ciascun tecnico
		\subitem Altro parametro di interesse nell'analisi dell'efficienza e della preparazione di un tecnico è il tempo impiegato in media per riparare i guasti
		che decide di voler sottoporre alla sua attenzione
	\item[\textbf{V4 -}] Distanza temporale media tra ricezione della chiamata e visita del tecnico per centro assistenza
		\subitem La "bontà" di un centro assistenza si può valutare sulla base di quanto rapidamente riesce a risolvere i vari interventi sottoposti. Si rende perciò
		utile poter calcolare per ogni centro assistenza la velocità media di risoluzione degli interventi.
\end{itemize}

\begin{tabularx}{\linewidth}{X|X|X}
	\hline
	\textbf{Codice operazione} & \textbf{Frequenza} & \textbf{Tipo (Interattiva / Batch)}\\
	\hline
	\hline
	O1 & 5 volte al giorno & I\\
	\hline
	O2 & 1 volta al mese & I\\
	\hline
	T1 & 1 volta al giorno & B\\
	\hline
	T2 & 5 volte al giorno & I\\
	\hline
	P1 & 5 volte al mese & B\\
	\hline
	P2 & 5 volte al mese & I\\
	\hline
	P3 & 5 volte al mese & B\\
	\hline
	P4 & 5 volte al mese & B\\
	\hline
	P5 & 5 volte al mese & B\\
	\hline
	P6 & 5 volte al mese & B\\
	\hline
	P7 & 5 volte al mese & B\\
	\hline
	P8 & 5 volte al mese & B\\
	\hline
	P9 & 5 volte al mese & B\\
	\hline
	P10 & 5 volte al mese & B\\
	\hline
	V1 & 1 volta al mese & B\\
	\hline
	V2 & 1 volta al mese & B\\
	\hline
	V3 & 1 volta al mese & B\\
	\hline
	V4 & 1 volta al mese & B\\
	\hline
	\caption{Tabella della frequenza delle principali operazioni}
\end{tabularx}

\section{Schemi di navigazione e tabelle degli accessi delle principali operazioni}

Per ogni operazione, verrà esplicitato a parole il modo con il quale verrà eseguita l'operazione e verrà mostrato uno schema che indichi come questi
accessi alle varie entità e relazioni si riflette sullo schema concettuale realizzato. Da questo schema, si estrarrà la tabella degli accessi che ci
permetterà di avere il costo totale di ognuna delle operazioni in termini di "\textit{I/O}".

\subsection{O1, O2 - Inserimento e cancellazione di un nuovo intervento}

L'apertura di ogni nuovo intervento coinciderà con l'aggiunta di un nuovo intervento e, nel caso peggiore, anche di un nuovo cliente che non aveva mai chiamato
prima d'ora e di uno o più nuovi prodotti che non si erano mai rotti. All'intervento viene inoltre associato l'operatore che lo registra.\newline
L'operazione di cancellazione è identica nelle operazioni, ne cambia semplicemente la frequenza.

\begin{tabularx}{\linewidth}{X|>{\hsize=0.4\hsize}X|X|>{\hsize=0.375\hsize}X}
	\hline
	\textbf{Costrutto\newline Coinvolto} & \textbf{Tipo\newline Costrutto} & \textbf{Accessi} & \textbf{Tipo\newline Accesso}\\
	\hline
	\hline
	Cliente & E & 1 & S\\
	\hline
	Richiesta & A & 1 & S\\
	\hline
	Intervento & E & 1 & S\\
	\hline
	Motivazione & A & 158.640 / 158.400 = 1,002 & S\\
	\hline
	Prodotto & E & 1,002 & S\\
	\hline
	Appartenenza & A & 1,002 & S\\
	\hline
	Proprietà & A & 1,002 & S\\
	\hline
	Effetto & A & 1,002 & S\\
	\hline
	Guasto & E & 1,002 & S\\
	\hline
	\hline
	\multicolumn{2}{>{\hsize=2\hsize}X|}{TOTALE} & \multicolumn{2}{>{\hsize=2\hsize}X}{ 18,024 accessi }\\\hline
	\hline
	\caption{Calcolo degli accessi delle operazioni O1 e O2}
\end{tabularx}

\begin{figure}[H]
	\centering
	\includegraphics[width=\linewidth]{images/O1-O2.png}
	\caption{Schema di navigazione per le operazioni O1 e O2}
\end{figure}

\newpage

\subsection{T1 - Visualizzazione degli interventi aperti}

Questa operazione visualizza tutti e soli gli interventi correntemente aperti, perciò i costrutti che coinvolge sono banali:
solamente l'entità intervento.

\begin{tabularx}{\linewidth}{X|>{\hsize=0.4\hsize}X|X|>{\hsize=0.375\hsize}X}
	\hline
	\textbf{Costrutto\newline Coinvolto} & \textbf{Tipo\newline Costrutto} & \textbf{Accessi} & \textbf{Tipo\newline Accesso}\\
	\hline
	\hline
	Intervento & E & 158.400 & L\\
	\hline
	\hline
	\multicolumn{2}{>{\hsize=2\hsize}X|}{TOTALE} & \multicolumn{2}{>{\hsize=2\hsize}X}{ 158.400 accessi }\\\hline
	\hline
	\caption{Calcolo degli accessi dell'operazione T1}
\end{tabularx}

\begin{figure}[H]
	\centering
	\includegraphics{images/T1.png}
	\caption{Schema di navigazione per l'operazione T1}
\end{figure}

\subsection{T2 - Chiusura di un intervento}

Si tratta di un'operazione svolta da un tecnico una volta giunto a casa di un cliente fatta dopo la riparazione dell'elettrodomestico del cliente. I dati necessariamente
aggiornati sono quelli relativi all'intervento e a ciascuno dei guasti riparati, quindi anche le informazioni che associano i guasti ai difetti, mentre quelli che potrebbero essere inseriti
in aggiunta nel caso peggiore sono quelli relativi al prodotto o ai prodotti che il cliente non è riuscito a reperire da solo. Per semplicità, assumiamo che tutti i difetti necessari siano
già stati preventivamente inseriti all'interno del \textit{database}.

\begin{tabularx}{\linewidth}{X|>{\hsize=0.4\hsize}X|X|>{\hsize=0.375\hsize}X}
	\hline
	\textbf{Costrutto\newline Coinvolto} & \textbf{Tipo\newline Costrutto} & \textbf{Accessi} & \textbf{Tipo\newline Accesso}\\
	\hline
	\hline
	Intervento & E & 1 & L\\
	\hline
	Intervento & E & 1 & S\\
	\hline
	Motivazione & A & 158.640 / 158.400 = 1,002 & L\\
	\hline
	Guasto & E & 1,002 & L\\
	\hline
	Guasto & E & 1,002 & S\\
	\hline
	Effetto & A & 1,002 & L\\
	\hline
	Prodotto & E & 1,002 & L\\
	\hline
	Prodotto & E & 1,002 & S\\
	\hline
	Causa & A & 1,002 & S\\
	\hline
	\hline
	\multicolumn{2}{>{\hsize=2\hsize}X|}{TOTALE} & \multicolumn{2}{>{\hsize=2\hsize}X}{ 13,020 accessi }\\\hline
	\hline
	\caption{Calcolo degli accessi dell'operazione T2}
\end{tabularx}

\begin{figure}[H]
	\centering
	\includegraphics[width=\linewidth]{images/T2.png}
	\caption{Schema di navigazione per l'operazione T2}
\end{figure}

\newpage

\subsection{P1 - Visualizzazione di tutti i dati inerenti ai guasti del mese corrente}

Per visualizzare tutti i dati inerenti a tutti guasti accaduti nel mese corrente, occorre passare in rassegna tutti gli elementi dell'entità "guasto" e da questi risalire ai dati
degli interventi a cui sono associati. Dopodiché, è necessario associare loro i prodotti che hanno
coinvolto e i difetti che li hanno causati con tanto di ricambi utilizzati. D'ora in poi, per questa e per tutte le operazioni successive, si supporrà che tutti i \textit{record}
nel \textit{database} siano equamente distribuiti tra i mesi, in modo tale che basti dividere per dodici per trovare il numero di record effettivamente interessati da un'operazione.
Supporremo di non sapere a priori quali sono i \textit{record} appartenenti a quale mese e perciò dovremo passare in rassegna preventivamente tutti i guasti.

\begin{tabularx}{\linewidth}{X|>{\hsize=0.4\hsize}X|X|>{\hsize=0.375\hsize}X}
	\hline
	\textbf{Costrutto\newline Coinvolto} & \textbf{Tipo\newline Costrutto} & \textbf{Accessi} & \textbf{Tipo\newline Accesso}\\
	\hline
	\hline
	Guasto & E & 158.640 & L\\
	\hline
	Motivazione & A & 158.640 / 12 = 13.220 & L\\
	\hline
	Intervento & E & 13.220 & L\\
	\hline
	Effetto & A & 13.220 & L\\
	\hline
	Prodotto & E & 13.220 & L\\
	\hline
	Causa & A & 13.220 & L\\
	\hline
	Difetto & E & 13.220 & L\\
	\hline
	Riparazione & A & 13.220 * (75 / 50) = 19.830 & L\\
	\hline
	Ricambio & E & 19.830 & L\\
	\hline
	\hline
	\multicolumn{2}{>{\hsize=2\hsize}X|}{TOTALE} & \multicolumn{2}{>{\hsize=2\hsize}X}{ 277.620 accessi }\\\hline
	\hline
	\caption{Calcolo degli accessi dell'operazione P1}
\end{tabularx}

\begin{figure}[H]
	\centering
	\includegraphics[width=\linewidth]{images/P1.png}
	\caption{Schema di navigazione per l'operazione P1}
\end{figure}

\subsection{P2, P3 - Visualizzazione di tutti i guasti che presentano una certa parola chiave e i primi cinque PNC maggiormente coinvolti nei guasti del mese}

Nell'operazione P2 si prendono in considerazione tutti i guasti e si mostrano soltanto quelli che, nella descrizione inserita dal tecnico, presentano una certa parola
chiave. Questo significa che l'unica entità coinvolta è "guasto" e lo schema di navigazione di P2 sarà perciò molto semplificato. Stesso discorso vale per l'operazione P3:
se si vuole raggruppare i guasti per PNC, e poi mostrare i cinque sottogruppi con il maggior numero di membri, l'unico modo per farlo è passare in rassegna tutti gli elementi
di "guasto". Potrebbe sembrare necessario l'accesso all'entità "prodotto" e all'entità "intervento", ma in questo caso, poiché l'attributo "PNC" è parte integrante della
chiave di "prodotto" e la data di richiesta dell'intervento è parte della chiave di "intervento", questi valori fanno parte della chiave di "guasto"
e perciò non sarà necessario nessun accesso ad entità ulteriori.

\newpage

\begin{tabularx}{\linewidth}{X|>{\hsize=0.4\hsize}X|X|>{\hsize=0.375\hsize}X}
	\hline
	\textbf{Costrutto\newline Coinvolto} & \textbf{Tipo\newline Costrutto} & \textbf{Accessi} & \textbf{Tipo\newline Accesso}\\
	\hline
	\hline
	Guasto & E & 158.640 & L\\
	\hline
	\hline
	\multicolumn{2}{>{\hsize=2\hsize}X|}{TOTALE} & \multicolumn{2}{>{\hsize=2\hsize}X}{ 158.640 accessi }\\\hline
	\hline
	\caption{Calcolo degli accessi delle operazioni P2 e P3}
\end{tabularx}

\begin{figure}[H]
	\centering
	\includegraphics{images/P2-P3.png}
	\caption{Schema di navigazione per le operazioni P2 e P3}
\end{figure}

\subsection{P4 - I primi cinque component code maggiormente presenti nei guasti di questo mese}

Anche questa operazione si inserisce sulla scia di P3, con la differenza che, dopo aver ottenuto tutti i guasti del mese corrente, li si raggruppa sulla base del
\textit{component code} del difetto che li ha causati, per poi contare quanti elementi sono presenti in ogni gruppo e infine ordinare i gruppi sulla base della quantità dei loro membri.

\begin{tabularx}{\linewidth}{X|>{\hsize=0.4\hsize}X|X|>{\hsize=0.375\hsize}X}
	\hline
	\textbf{Costrutto\newline Coinvolto} & \textbf{Tipo\newline Costrutto} & \textbf{Accessi} & \textbf{Tipo\newline Accesso}\\
	\hline
	\hline
	Guasto & E & 158.640 & L\\
	\hline
	Causa & A & 158.640 / 12 = 13.220 & L\\
	\hline
	Difetto & E & 13.220 & L\\
	\hline
	\hline
	\multicolumn{2}{>{\hsize=2\hsize}X|}{TOTALE} & \multicolumn{2}{>{\hsize=2\hsize}X}{ 185.080 accessi }\\\hline
	\hline
	\caption{Calcolo degli accessi dell'operazione P4}
\end{tabularx}

\begin{figure}[H]
	\centering
	\includegraphics{images/P4.png}
	\caption{Schema di navigazione per l'operazione P4}
\end{figure}

\subsection{P5, P6 - I primi cinque ricambi più utilizzati nel riparare i guasti del mese corrente e gli interventi più costosi per le riparazioni}

L'operazione P5 non si discosta molto dalle precedenti perché si occupa, una volta ottenuti i guasti del mese, di associarli ai difetti che li hanno causati e poi questi
ai ricambi necessari per ripararli. Una volta fatto, li si raggruppa per ricambio utilizzato e si prendono i primi cinque gruppi che presentano il maggior numero di membri.
L'operazione P6 invece raggruppa i guasti per singolo intervento e ne calcola il costo, scegliendo poi i primi cinque gruppi che presentano il costo dato dalla riparazione maggiore
di quello degli altri. In questo caso si tiene conto del numero dei ricambi utilizzati, ma anche del loro costo unitario.

\newpage

\begin{tabularx}{\linewidth}{X|>{\hsize=0.4\hsize}X|X|>{\hsize=0.375\hsize}X}
	\hline
	\textbf{Costrutto\newline Coinvolto} & \textbf{Tipo\newline Costrutto} & \textbf{Accessi} & \textbf{Tipo\newline Accesso}\\
	\hline
	\hline
	Guasto & E & 158.640 & L\\
	\hline
	Causa & A & 158.640 / 12 = 13.220 & L\\
	\hline
	Difetto & E & 13.220 & L\\
	\hline
	Riparazione & A & 13.220 & L\\
	\hline
	Ricambio & E & (75 / 50) * 13.220 = 19.830  & L\\
	\hline
	\hline
	\multicolumn{2}{>{\hsize=2\hsize}X|}{TOTALE} & \multicolumn{2}{>{\hsize=2\hsize}X}{ 218.130 accessi }\\
	\hline
	\hline
	\caption{Calcolo degli accessi delle operazioni P5 e P6}
\end{tabularx}

\begin{figure}[H]
	\centering
	\includegraphics[width=\linewidth]{images/P5-P6.png}
	\caption{Schema di navigazione per le operazioni P5 e P6}
\end{figure}

\newpage

\subsection{P7 - Le prime cinque zone per numero di guasti di questo mese}

In questa operazione, come nelle precedenti, si prendono in considerazione i guasti del mese corrente, per poi risalire sempre tramite agli interventi che contengono i guasti,
agli operatori che hanno registrato gli interventi e infine attraverso questi ai centri assistenza che hanno accolto la richiesta, identificando l'area di competenza da cui proviene
l'intervento. A questo punto, non resta che raggruppare per area e visualizzare i primi cinque gruppi per numero di membri.

\begin{tabularx}{\linewidth}{X|>{\hsize=0.4\hsize}X|X|>{\hsize=0.375\hsize}X}
	\hline
	\textbf{Costrutto\newline Coinvolto} & \textbf{Tipo\newline Costrutto} & \textbf{Accessi} & \textbf{Tipo\newline Accesso}\\
	\hline
	Guasto & E & 158.640 & L\\
	\hline
	Motivazione & A & 158.640 / 12 = 13.220 & L\\
	\hline
	Intervento & E & 13.220 & L\\
	\hline
	Apertura & A & 13.220 & L\\
	\hline
	Operatore & E & 13.220 & L\\
	\hline
	Afferenza\_Op & A & 13.220 & L\\
	\hline
	Centro Assistenza & E & 13.220 & L\\
	\hline
	\hline
	\multicolumn{2}{>{\hsize=2\hsize}X|}{TOTALE} & \multicolumn{2}{>{\hsize=2\hsize}X}{ 237.960 accessi }\\\hline
	\hline
	\caption{Calcolo degli accessi dell'operazione P7}
\end{tabularx}

\begin{figure}[H]
	\centering
	\includegraphics[width=\linewidth]{images/P7.png}
	\caption{Schema di navigazione per l'operazione P7}
\end{figure}

\subsection{P8, P9 - I "Time To Failure" per ciascun prodotto a partire dalla data di acquisto e dalla data di installazione}

Per le due operazioni P8 e P9 ci si preoccupa di poter estrarre i cosiddetti "Time To Failure" di ciascun prodotto, cioè il
tempo che è passato tra una certa data fissata e il momento in cui il prodotto ha subito il guasto. L'operazione P8 fissa il momento iniziale alla data di acquisto
del prodotto, mentre l'operazione P9 alla data di installazione dello stesso. Per eseguire entrambe queste due operazioni, partendo dai guasti del mese corrente, si
sceglie come data di rottura dell'elettrodomestico la data di apertura dell'intervento associato, che abbiamo disponibile senza accessi aggiuntivi, e poi si risale
al prodotto associato da cui si estrae l'altra data di interesse.

\begin{tabularx}{\linewidth}{X|>{\hsize=0.4\hsize}X|X|>{\hsize=0.375\hsize}X}
	\hline
	\textbf{Costrutto\newline Coinvolto} & \textbf{Tipo\newline Costrutto} & \textbf{Accessi} & \textbf{Tipo\newline Accesso}\\
	\hline
	\hline
	Guasto & E & 158.640 & L\\
	\hline
	Effetto & A & 158.640 / 12 = 13.220 & L\\
	\hline
	Prodotto & E & 13.220 & L\\
	\hline
	\hline
	\multicolumn{2}{>{\hsize=2\hsize}X|}{TOTALE} & \multicolumn{2}{>{\hsize=2\hsize}X}{ 185.080 accessi }\\\hline
	\hline
	\caption{Calcolo degli accessi delle operazioni P8 e P9}
\end{tabularx}

\begin{figure}[H]
	\centering
	\includegraphics{images/P8-P9.png}
	\caption{Schema di navigazione per le operazioni P8 e P9}
\end{figure}

\subsection{P10 - Tempo medio di riparazione di un guasto per tipo di difetto}

Per poter calcolare il tempo medio di riparazione di un guasto occorre per prima cosa accedere all'intervento che lo contiene ed estrarre da questo il tempo impiegato per ripararlo. Dopodiché, per ogni guasto è necessario accedere al difetto che lo ha causato così da poter raggruppare i guasti per difetto.

\begin{tabularx}{\linewidth}{X|>{\hsize=0.4\hsize}X|X|>{\hsize=0.375\hsize}X}
	\hline
	\textbf{Costrutto\newline Coinvolto} & \textbf{Tipo\newline Costrutto} & \textbf{Accessi} & \textbf{Tipo\newline Accesso}\\
	\hline
	\hline
	Guasto & E & 158.640 & L\\
	\hline
	Motivazione & A & 158.640 & L\\
	\hline
	Intervento & E & 158.640 & L\\
	\hline
	Causa & A & 158.640 & L\\
	\hline
	Difetto & E & 158.640 & L\\
	\hline
	\hline
	\multicolumn{2}{>{\hsize=2\hsize}X|}{TOTALE} & \multicolumn{2}{>{\hsize=2\hsize}X}{ 793.200 accessi }\\\hline
	\hline
	\caption{Calcolo degli accessi dell'operazione P10}
\end{tabularx}

\begin{figure}[H]
	\centering
	\includegraphics[width=\linewidth]{images/P10.png}
	\caption{Schema di navigazione per l'operazione P10}
\end{figure}

\subsection{V1 - Conteggio di tutti gli interventi aperti nel mese da ciascun operatore}

In questa operazione si suddividono gli interventi che sono stati fatti in questo mese tra i vari operatori sulla base di chi ha aperto quale intervento. Si effettua
poi un raggruppamento per operatore e si contano gli elementi per ciascun sottogruppo.

\begin{tabularx}{\linewidth}{X|>{\hsize=0.4\hsize}X|X|>{\hsize=0.375\hsize}X}
	\hline
	\textbf{Costrutto\newline Coinvolto} & \textbf{Tipo\newline Costrutto} & \textbf{Accessi} & \textbf{Tipo\newline Accesso}\\
	\hline
	\hline
	Intervento & E & 158.400 & L\\
	\hline
	Apertura & A & 158.400 / 12 = 13.200 & L\\
	\hline
	Operatore & E & 13.200 & L\\
	\hline
	\hline
	\multicolumn{2}{>{\hsize=2\hsize}X|}{TOTALE} & \multicolumn{2}{>{\hsize=2\hsize}X}{ 184.800 accessi }\\\hline
	\hline
	\caption{Calcolo degli accessi dell'operazione V1}
\end{tabularx}

\begin{figure}[H]
	\centering
	\includegraphics{images/V1.png}
	\caption{Schema di navigazione per l'operazione V1}
\end{figure}

\subsection{V2, V3 - Conteggio di tutti gli interventi chiusi nel mese e tempo medio di riparazione per ciascun tecnico}

Queste due operazioni non sono per nulla dissimili dalla precedente, se non per il fatto che i dipendenti coinvolti sono i tecnici e non gli operatori e,
se l'operazione V2 è identica nei passaggi alla precedente, l'operazione V3 sceglie come criterio da visualizzare per ciascun gruppo non tanto il numero di
elementi dello stesso ma la media dei tempi necessari per la risoluzione dei problemi che il tecnico ha sistemato.

\begin{tabularx}{\linewidth}{X|>{\hsize=0.4\hsize}X|X|>{\hsize=0.375\hsize}X}
	\hline
	\textbf{Costrutto\newline Coinvolto} & \textbf{Tipo\newline Costrutto} & \textbf{Accessi} & \textbf{Tipo\newline Accesso}\\
	\hline
	\hline
	Intervento & E & 158.400 & L\\
	\hline
	Assistenza & A & 158.400 / 12 = 13.200 & L\\
	\hline
	Tecnico & E & 13.200 & L\\
	\hline
	\hline
	\multicolumn{2}{>{\hsize=2\hsize}X|}{TOTALE} & \multicolumn{2}{>{\hsize=2\hsize}X}{ 184.800 accessi }\\\hline
	\hline
	\caption{Calcolo degli accessi delle operazioni V2 e V3}
\end{tabularx}

\begin{figure}[H]
	\centering
	\includegraphics{images/V2-V3.png}
	\caption{Schema di navigazione per le operazioni V2 e V3}
\end{figure}

\subsection{V4 -  Distanza temporale media tra ricezione della chiamata e visita del tecnico per centro assistenza}

Questa operazione raggruppa gli interventi per centro assistenza che li ha creati e per ogni gruppo calcola la media del tempo di vita di un intervento, dalla sua apertura
dopo una chiamata alla sua chiusura dopo la visita di un tecnico.

\begin{tabularx}{\linewidth}{X|>{\hsize=0.4\hsize}X|X|>{\hsize=0.375\hsize}X}
	\hline
	\textbf{Costrutto\newline Coinvolto} & \textbf{Tipo\newline Costrutto} & \textbf{Accessi} & \textbf{Tipo\newline Accesso}\\
	\hline
	\hline
	Intervento & E & 158.400 & L\\
	\hline
	Apertura & A & 158.400 & L\\
	\hline
	Operatore & E & 158.400 & L\\
	\hline
	Afferenza\_Op & A & 158.400 & L\\
	\hline
	Centro Assistenza & E & 158.400 & L\\
	\hline
	\hline
	\multicolumn{2}{>{\hsize=2\hsize}X|}{TOTALE} & \multicolumn{2}{>{\hsize=2\hsize}X}{ 792.000 accessi }\\\hline
	\hline
	\caption{Calcolo degli accessi dell'operazione V4}
\end{tabularx}

\begin{figure}[H]
	\centering
	\includegraphics{images/V4.png}
	\caption{Schema di navigazione per l'operazione V4}
\end{figure}

\newpage

\section{Raffinamento dello schema concettuale}

Per rendere traducibile lo schema "\textit{Entity-Relationship}" in uno schema relazionale, occorre effettuare alcune modifiche allo schema stesso
cercando di mantenere quanto più possibile alta la fedeltà dei concetti rappresentati.

\subsection{Eliminazione delle gerarchie}

La rimozione della gerarchia che discende direttamente da persona è, delle due presenti, la più semplice. L'entità persona, volta esclusivamente a fattorizzare
i campi comuni, non ha nessuna utilità pratica oltre quella citata all'interno dello schema. Si procede perciò ad eliminarla con un collasso verso il basso,
duplicando i campi "Nome" e "Cognome" per le entità "cliente" e "dipendente". Per la seconda gerarchia, quella che origina da "dipendente", si decide anche in questo
caso di effettuare un collasso verso il basso perché, osservando le principali operazioni individuate, nessuna richiede l'accesso ai dipendenti nel loro assieme,
anzi, al contrario. Ben tre operazioni vengono effettuate su una specifica categoria di dipendenti in maniera indipendente dagli altri e se decidessimo di creare un'unica
relazione con tutti i dipendenti finirebbero per aumentare il loro costo in termini di "\textit{I/O}". Inoltre, in entrambe le gerarchie, non abbiamo nessun tipo di associazione
che figura legata all'entità padre, perciò nessuno vi farà mai accesso e non saranno presenti dopo la modifica associazioni duplicate. Quello che succede è che però
vengono persi alcuni vincoli che vanno implementati a parte, ovvero che l'unicità degli attributi "CF" e "Codice" deve valere non solo all'interno delle singole
relazioni, ma anche tra tutte e tre le relazioni che discendono da "dipendente".

\subsection{Eliminazione di attributi multipli o composti}

Nello schema non sono presenti attributi multipli, ma sono presenti attributi composti. Poiché il loro scopo era semplicemente segnalare il fatto che quegli attributi erano date
che inglobavano non solo la data in sè e per sè, ma anche l'ora, possiamo semplicemente eliminarli senza problemi, avendo però cura di scegliere il tipo di dato adeguato
per quegli attributi una volta che creeremo la relazione corrispondente.

\subsection{Scelta delle chiavi primarie}

Lo schema "\textit{Entity-Relationship}" evidenzia già in maniera chiara gli attributi che concorrono ad essere potenziali chiavi primarie e si decide di utilizzare quelle
indicate senza effettuare ulteriori modifiche. Molte chiavi infatti sono già dei semplici codici e perciò non è necessario semplificarle. Le chiavi più complesse che appaiono nello
schema sono quelle di "intervento" e "guasto". La prima entità è individuata da un "cliente" e da una data. "Cliente" però è identificato univocamente dal solo numero di telefono,
facendo perciò in modo che un intervento sia identificato da una data e un numero di telefono. Conseguentemente, un guasto viene identificato da un numero di telefono, una data e i
due codici identificativi di prodotto. Entrambi questi due identificatori perciò non appaiono essere troppo complessi e in particolar modo l'identificatore di guasto presenta alcuni
attributi - "DataIntervento", "PNC" - utilizzati in alcune operazioni che coinvolgono i guasti, permettendo di averli già disponibili senza effettuare ulteriori accessi ad altre relazioni,
diminuendo i costi legati a quelle specifiche operazioni.

\begin{figure}[H]
	\centering
	\includegraphics[width=\linewidth]{images/refined.png}
	\caption{Schema E/R dopo i raffinamenti individuati nei paragrafi precedenti}
\end{figure}

\newpage

\section{Analisi delle ridondanze}

Le associazioni che figurano come ridondanti nello schema sono due: "analisi", tra "progettista" e "guasto", e "proprietà", tra "cliente" e "prodotto". Queste due associazioni
condividono il fatto che nessuna delle due è utilizzata delle principali operazioni individuate e in particolar modo, nessuna delle due è particolarmente utile ai nostri scopi.\newline
L'associazione "analisi", infatti, dovrebbe tenere conto di tutti i possibili guasti che sono di interesse per ciascuno dei progettisti, che però provocherebbe non poche ridondanze
a livello di record per nessun utilizzo. Vero è che abbiamo indicato che non tutti i progettisti sono interessati a visionare tutti i guasti, ma filtrare questi dati può essere effettuato
anche senza la presenza esplicita di questa associazione, tanto più che abbiamo indicato le operazioni di creazione delle statistiche come "\textit{batch}", quindi da eseguire in
un momento diverso, possibilmente antecedente, rispetto a quello in cui i progettisti accederanno ai dati. Sarà poi quando un progettista richiederà la visualizzazione dei dati che
essi saranno filtrati sulla base delle categorie di suo interesse.\newline
Per quanto riguarda invece "proprietà", pur essendo vero che il cliente effettua una telefonata per un elettrodomestico che possiede, non abbiamo alcun interesse ad utilizzare mai
questa nozione. Il fatto che sia presente il cliente associato ad un intervento è perché necessitiamo di sapere presso chi mandare il tecnico per la riparazione. Quale che sia il prodotto
che possiede è a noi indifferente, poiché non ci interessano statistiche di vendita o similari. Nell'estremo caso volessimo risalire, per un dato cliente, a quali elettrodomestici
ha comprato - e gli si sono necessariamente guastati - possiamo sempre farlo attraverso "intervento" e "guasto".\newline
Attributi ridondanti eliminabili già presenti nello schema "\textit{Entity-Relationship}" non sono presenti. Per semplificare le operazioni da effettuare su base mensile, si può pensare di importare la data di apertura dell'intervento all'interno dell'entità "guasto", cosa che però viene già effettuata in quanto "Intervento" è parte integrante della chiave di "guasto" e perciò
non è richiesta alcuna modifica ulteriore. Come visto in precedenza, lo stesso discorso è applicabile anche all'attributo "\textit{PNC}" di "prodotto" e "guasto".\newline
Inoltre, per mantenere i costi di accesso delle operazioni uguali a quelli individuati nella sezione dedicata, dobbiamo poter permettere che si possa immediatamente
individuare se un dato guasto è utile ad un dato progettista, perché ricade nella sua categoria di interesse, oppure no, senza ulteriori costi aggiuntivi. Questo può essere fatto
copiando il valore dell'attributo "Codice" dell'istanza dell'entità "categoria" a cui il prodotto associato al guasto interessato è a sua volta associato. Non abbiamo nessuna
variazione dei costi per l'inserimento nel caso peggiore, restano identici a come li avevamo calcolati, perciò abbiamo un deciso guadagno in termini di numero di accessi.\newline
Un'altra modifica che si può effettuare è importare l'area in cui l'intervento viene aperto all'interno dell'entità "intervento" stessa. Questo attributo "Zona" sarebbe perciò
l'area di competenza del centro assistenza a cui l'operatore che riceve la chiamata fa capo, in quanto si suppone che ogni cliente che chiama venga diretto al centro assistenza 
adeguato alla sua zona di residenza. Questa introduzione di ridondanza porterebbe all'aumento del costo dell'operazione O1, ma ad una diminuzione del costo di P7.

\captionsetup[table]{skip=-10pt}
\begin{table}[H]
	\caption*{\textbf{Operazione O1 con ridondanza}}
	\begin{tabularx}{\linewidth}{X|>{\hsize=0.4\hsize}X|X|>{\hsize=0.375\hsize}X}
		\hline
		\textbf{Costrutto\newline Coinvolto} & \textbf{Tipo\newline Costrutto} & \textbf{Accessi} & \textbf{Tipo\newline Accesso}\\
		\hline
		\hline
		Afferenza\_Op & A & 1 & L\\
		\hline
		Centro\_Assistenza & E & 1 & L\\
		\hline
		Cliente & E & 1 & S\\
		\hline
		Richiesta & A & 1 & S\\
		\hline
		Intervento & E & 1 & S\\
		\hline
		Motivazione & A & 158.640 / 158.400 = 1,002 & S\\
		\hline
		Guasto & E & 1,002 & S\\
		\hline
		Effetto & A & 1,002 & S\\
		\hline
		Prodotto & E & 1,002 & S\\
		\hline
		Appartenenza & A & 1,002 & S\\
		\hline
		Proprietà & A & 1,002 & S\\
		\hline
		\multicolumn{2}{>{\hsize=2\hsize}X|}{TOTALE} & \multicolumn{2}{>{\hsize=2\hsize}X}{ 20,024 accessi }\\
		\hline
		\hline
		\multicolumn{2}{>{\hsize=2\hsize}X|}{TOTALE MENSILE} & \multicolumn{2}{>{\hsize=2\hsize}X}{ 20,024 * 5 accessi/giorno * 30 giorno/mese = 3.003,600 accessi/mese }\\
		\hline
		\hline
	\end{tabularx}
	\caption*{\textbf{Operazione P7 con ridondanza}}
	\begin{tabularx}{\linewidth}{X|>{\hsize=0.4\hsize}X|X|>{\hsize=0.375\hsize}X}
		\hline
		\textbf{Costrutto\newline Coinvolto} & \textbf{Tipo\newline Costrutto} & \textbf{Accessi} & \textbf{Tipo\newline Accesso}\\
		\hline
		\hline
		Guasto & E & 158.640 & L\\
		\hline
		Motivazione & A & 158.640 / 12 = 13.220 & L\\
		\hline
		Intervento & E & 13.220 & L\\
		\hline
		\hline
		\multicolumn{2}{>{\hsize=2\hsize}X|}{TOTALE} & \multicolumn{2}{>{\hsize=2\hsize}X}{ 185.080 accessi }\\
		\hline
		\hline
		\multicolumn{2}{>{\hsize=2\hsize}X|}{TOTALE MENSILE} & \multicolumn{2}{>{\hsize=2\hsize}X}{ 185.080 * 5 accessi/mese = 925.400 accessi/mese }\\
		\hline
		\hline
	\end{tabularx}
\end{table}
\captionsetup[table]{skip=10pt}

\newpage

Perciò, abbiamo che il numero totale di accessi mensili supponendo di mantenere la ridondanza è di 928.403,600 contro i 1.192.503,600 accessi effettuati in caso di mancanza della
ridondanza, rendendo utile il suo mantenimento.

\begin{figure}[H]
	\centering
	\includegraphics[width=\linewidth]{images/Unredundant.png}
	\caption{Schema E/R ulteriormente raffinato in seguito all'analisi delle ridondanze}
\end{figure}

\section{Traduzione di entità e associazioni in relazioni}

\begin{itemize}
	\item \textbf{CENTRI\_ASSISTENZA} (\underline{CodiceNazionale}, \underline{Nazione}, Sede, AreaCompetenza)
	\item \textbf{OPERATORI} (\underline{Codice}, CF, Nome, Cognome, DataNascita, LuogoNascita, Residenza, Stipendio, CodiceNazionaleCentro, NazioneCentro)
		\begin{itemize}[leftmargin=*, topsep=0pt]
			\item[] FK: (CodiceNazionaleCentro, NazioneCentro) REFERENCES CENTRI\_ASSISTENZA
			\item[] Unique(CF)
		\end{itemize}
	\item \textbf{TECNICI} (\underline{Codice}, Nome, Cognome, CF,  DataNascita, LuogoNascita, Residenza, Stipendio, DipendenteInterno, CodiceNazionaleCentro, NazioneCentro)
		\begin{itemize}[leftmargin=*, topsep=0pt]
			\item[] FK: (CodiceNazionaleCentro, NazioneCentro) REFERENCES CENTRI\_ASSISTENZA
			\item[] Unique(CF)
		\end{itemize}
	\item \textbf{CLIENTI} (\underline{NumeroTelefono}, Nome, Cognome, Recapito, Email*)
	\item \textbf{INTERVENTI} (\underline{NumeroTelefonoCliente}, \underline{DataRichiesta}, Stato, DataVisita, TempoImpiegato*, Zona, CodiceOperatore, CodiceTecnico*)
		\begin{itemize}[leftmargin=*, topsep=0pt]
			\item[] FK: (NumeroTelefonoCliente) REFERENCES CLIENTI
			\item[] FK: (CodiceOperatore) REFERENCES OPERATORI
			\item[] FK: (CodiceTecnico) REFERENCES TECNICI
		\end{itemize}		
	\item \textbf{CATEGORIE} (\underline{Codice}, Nome)
	\item \textbf{PRODOTTI} (\underline{PNC}, \underline{SNC}, Modello, DataAcquisto*, DataInstallazione*, CodiceGaranzia*, CodiceCategoria)
		\begin{itemize}[leftmargin=*, topsep=0pt]
			\item[] FK: (CodiceCategoria) REFERENCES CATEGORIE
		\end{itemize}		
	\item \textbf{DIFETTI} (\underline{ComponentCode}, \underline{CodiceTipo}, NomeComponente, NomeTipo)
	\item \textbf{RICAMBI} (\underline{Codice}, Nome, CostoAcquisto, CostoInstallazione, ComponentCode, CodiceTipoDifetto)
		\begin{itemize}[leftmargin=*, topsep=0pt]
			\item[] FK: (ComponentCode, CodiceTipoDifetto) REFERENCES DIFETTI
		\end{itemize}
	\newpage	
	\item \textbf{GUASTI} (\underline{NumeroTelefonoCliente}, \underline{DataRichiestaIntervento}, \underline{PNC}, \underline{SNC}, DescrizioneCliente, DescrizioneTecnico*, CategoriaProdotto, ComponentCode*,
	CodiceTipoDifetto*)
		\begin{itemize}[leftmargin=*, topsep=0pt]
			\item[] FK: (NumeroTelefonoCliente, DataRichiestaIntervento) REFERENCES INTERVENTI
			\item[] FK: (PNC, SNC) REFERENCES PRODOTTI
			\item[] FK: (ComponentCode*, CodiceTipoDifetto*) REFERENCES DIFETTI
		\end{itemize}
	\item \textbf{PROGETTISTI} (\underline{Codice}, Nome, Cognome, CF, DataNascita, LuogoNascita, Residenza, Stipendio)
		\begin{itemize}[leftmargin=*, topsep=0pt]
			\item[] Unique(CF)
		\end{itemize}
	\item \textbf{INTERESSI} (\underline{CodiceCategoria}, \underline{CodiceProgettista})
		\begin{itemize}[leftmargin=*, topsep=0pt]
			\item[] FK: (CodiceCategoria) REFERENCES CATEGORIE
			\item[] FK: (CodiceProgettista) REFERENCES PROGETTISTI
		\end{itemize}
\end{itemize}

\section{Schema relazionale finale}

\begin{figure}[H]
	\centering
	\includegraphics[width=\linewidth]{images/logic.png}
	\caption{Schema logico relazionale ottenuto dal paragrafo precedente}
\end{figure}

\newpage

\section{Traduzione delle operazioni in SQL}

\begin{itemize}
	\item[\textbf{O1} -] Inserimento di un nuovo intervento
		\begin{itemize}[leftmargin=*, topsep=0pt]
			\item INSERT INTO Clienti (NumeroTelefono, Nome, Cognome, Recapito, Email) \\VALUES (?, ?, ?, ?, ?);
				\subitem (In caso il cliente non fosse presente)\newline
			\item INSERT INTO Prodotti (PNC, SNC, Modello, DataAcquisto, DataInstallazione, CodiceGaranzia, CodiceCategoria) \\VALUES (?, ?, ?, ?, ?, ?, ?);
				\subitem (In caso il prodotto non fosse presente)\newline
			\item INSERT INTO Interventi (NumeroTelefonoCliente, DataRichiesta, Stato, DataVisita, TempoImpiegato, Zona, CodiceOperatore, CodiceTecnico) \\VALUES (?, ?, ?, ?, ?, ?, ?, ?);\newline			
			\item INSERT INTO Guasti(NumeroTelefonoCliente, DataRichiestaIntervento, PNC, SNC, DescrizioneCliente, DescrizioneTecnico, CategoriaProdotto, ComponentCode, CodiceTipoDifetto) \\VALUES (?, ?, ?, ?, ?, ?, ?, ?, ?);
				\subitem (Per ogni guasto contenuto nell'intervento inserito)
		\end{itemize}
	\item[\textbf{O2} -] Cancellazione di un intervento precedentemente aperto dall'operatore
		\begin{itemize}[leftmargin=*, topsep=0pt]
			\item DELETE FROM Guasti \\WHERE Guasti.NumeroTelefonoCliente = ? AND Guasti.DataRichiestaIntervento = ? AND Guasti.PNC = ? AND Guasti.SNC = ?;
				\subitem (Per ognuno dei guasti inseriti con questo intervento)\newline
			\item DELETE FROM Interventi \\WHERE Interventi.NumeroTelefonoCliente = ? AND Interventi.DataRichiesta = ?;\newline
			\item DELETE FROM Clienti \\WHERE Clienti.NumeroTelefono = ?;
				\subitem (In caso avessimo inserito un cliente \textit{ex-novo})\newline
			\item DELETE FROM Prodotti \\WHERE Prodotti.PNC = ? AND Prodotti.SNC = ?;
				\subitem (In caso avessimo inserito un prodotto \textit{ex-novo})			
		\end{itemize}
	\item[\textbf{T1 -}] Vista di tutti gli interventi correntemente aperti non già assegnati ad un altro tecnico, ordinati per urgenza
		\begin{itemize}[leftmargin=*, topsep=0pt]
			\item SELECT Interventi.NumeroTelefonoCliente, Interventi.DataRichiesta\\
			FROM Interventi\\
			WHERE Interventi.CodiceTecnico IS NULL\\
			ORDER BY Interventi.DataRichiesta;
		\end{itemize}
	\item[\textbf{T2 -}] Chiusura di un intervento
		\begin{itemize}[leftmargin=*, topsep=0pt]
			\item UPDATE Interventi SET Interventi.CodiceTecnico = ?, Interventi.Stato = 'C', Interventi.TempoImpiegato = ? FROM Interventi WHERE Interventi.DataRichiesta = ? AND Interventi.NumeroTelefonoCliente = ?;\newline
			\item UPDATE Guasti SET Guasti.DescrizioneTecnico = ?, Guasti.CodiceTipoDifetto = ?, Guasti.ComponentCode = ? FROM Guasti WHERE Guasti.DataRichiestaIntervento = ? AND Guasti.NumeroTelefonoCliente = ? AND Guasti.PNC = ? AND Guasti.SNC = ?;\newline
			\item UPDATE Prodotti SET Prodotti.DataAcquisto = ?, Prodotti.DataInstallazione = ?, Prodotti.CodiceGaranzia = ? FROM Prodotti WHERE Prodotti.PNC = ? AND Prodotti.SNC = ?;
				\subitem (Nel caso avesse aiutato il cliente a ritrovare alcune informazioni sul prodotto, le informazioni ignote manterranno il campo con valore "NULL")
		\end{itemize}
	\newpage
	\item[\textbf{P1 -}] Visualizzare tutti i dati inerenti ai guasti accaduti nel mese corrente
		\begin{itemize}[leftmargin=*, topsep=0pt]
			\item SELECT Guasti.CategoriaProdotto, Guasti.PNC, Guasti.SNC, Prodotti.Modello, Guasti.DataRichiestaIntervento, Interventi.DataVisita, Prodotti.DataAcquisto, Prodotti.DataInstallazione, Interventi.Zona, Guasti.DescrizioneCliente, Guasti.DescrizioneTecnico, Guasti.ComponentCode, Difetti.NomeComponente, Guasti.CodiceTipoDifetto, Difetti.NomeTipo AS 'NomeTipoDifetto', Ricambi.Codice AS 'CodiceRicambio', Ricambi.Nome AS 'NomeRicambio', Ricambi.CostoAcquisto, Ricambi.CostoInstallazione\\
			FROM (((Guasti JOIN Interventi ON Guasti.NumeroTelefonoCliente = Interventi.NumeroTelefonoCliente AND Guasti.DataRichiestaIntervento = Interventi.DataRichiesta) JOIN Prodotti ON Guasti.PNC = Prodotti.PNC AND Guasti.SNC = Prodotti.SNC) JOIN Difetti ON Guasti.ComponentCode = Difetti.ComponentCode AND Guasti.CodiceTipoDifetto = Difetti.CodiceTipo) LEFT OUTER JOIN Ricambi ON Ricambi.ComponentCode = Difetti.ComponentCode AND Ricambi.CodiceTipoDifetto = Difetti.CodiceTipo\\
			WHERE MONTH(Guasti.DataRichiestaIntervento) = MONTH(CURRENT\_TIMESTAMP) AND YEAR(Guasti.DataRichiestaIntervento) = YEAR(CURRENT\_TIMESTAMP);
		\end{itemize}
	\item[\textbf{P2 -}] Visualizzare tutti i guasti che presentano una certa parola chiave nella descrizione scritta dal tecnico
		\begin{itemize}[leftmargin=*, topsep=0pt]
			\item SELECT Guasti.CategoriaProdotto, Guasti.PNC, Guasti.SNC, Guasti.DescrizioneCliente, Guasti.DescrizioneTecnico, Guasti.ComponentCode, Guasti.CodiceTipoDifetto\\
			FROM Guasti\\
			WHERE Guasti.CodiceTipoDifetto IS NOT NULL AND Guasti.ComponentCode IS NOT NULL AND MONTH(Guasti.DataRichiestaIntervento) = \\MONTH(CURRENT\_TIMESTAMP) AND YEAR(Guasti.DataRichiestaIntervento) = YEAR(CURRENT\_TIMESTAMP) AND CHARINDEX(?, Guasti.DescrizioneTecnico) $>$ 0;
		\end{itemize}
	\item[\textbf{P3 -}] I primi cinque \textit{PNC} che hanno subito più guasti nel mese corrente
		\begin{itemize}[leftmargin=*, topsep=0pt]
			\item SELECT TOP(5) WITH TIES Guasti.CategoriaProdotto, Guasti.PNC, COUNT(*) AS '\#'\\
			FROM Guasti\\
			WHERE Guasti.ComponentCode IS NOT NULL AND Guasti.CodiceTipoDifetto IS NOT NULL AND MONTH(Guasti.DataRichiestaIntervento) = \\MONTH(CURRENT\_TIMESTAMP) AND YEAR(Guasti.DataRichiestaIntervento) = YEAR(CURRENT\_TIMESTAMP)\\
			GROUP BY Guasti.CategoriaProdotto, Guasti.PNC ORDER BY 3 DESC;
		\end{itemize}
	\item[\textbf{P4 -}] I primi cinque \textit{Component Code} che sono stati coinvolti maggiormente nei guasti di questo mese
		\begin{itemize}[leftmargin=*, topsep=0pt]
			\item SELECT TOP(5) WITH TIES Guasti.CategoriaProdotto, Guasti.ComponentCode, Difetti.NomeComponente, COUNT(*) AS '\#'\\
			FROM Guasti JOIN Difetti ON Guasti.CodiceTipoDifetto = Difetti.CodiceTipo AND Guasti.ComponentCode = Difetti.ComponentCode\\
			WHERE MONTH(Guasti.DataRichiestaIntervento) = MONTH(CURRENT\_TIMESTAMP) AND YEAR(Guasti.DataRichiestaIntervento) = YEAR(CURRENT\_TIMESTAMP)\\
			GROUP BY Guasti.CategoriaProdotto, Guasti.ComponentCode, Difetti.NomeComponente ORDER BY 4 DESC;
		\end{itemize}
	\item[\textbf{P5 -}] I primi cinque ricambi più utilizzati per riparare i guasti di questo mese
		\begin{itemize}[leftmargin=*, topsep=0pt]
			\item SELECT TOP(5) WITH TIES Guasti.CategoriaProdotto, Ricambi.Codice, Ricambi.Nome, Ricambi.CostoAcquisto, Ricambi.CostoInstallazione, COUNT(*) AS '\#'\\
			FROM (Guasti JOIN Difetti ON Guasti.ComponentCode = Difetti.ComponentCode AND Guasti.CodiceTipoDifetto = Difetti.CodiceTipo) JOIN Ricambi ON Difetti.ComponentCode = Ricambi.ComponentCode AND Difetti.CodiceTipo = Ricambi.CodiceTipoDifetto\\
			WHERE MONTH(Guasti.DataRichiestaIntervento) = MONTH(CURRENT\_TIMESTAMP) AND YEAR(Guasti.DataRichiestaIntervento) = YEAR(CURRENT\_TIMESTAMP)\\
			GROUP BY Ricambi.Codice, Ricambi.Nome, Ricambi.CostoAcquisto, Ricambi.CostoInstallazione, Guasti.CategoriaProdotto ORDER BY 6 DESC;
		\end{itemize}
	\item[\textbf{P6 -}] I primi cinque interventi più costosi a partire dai ricambi nel mese corrente
		\begin{itemize}[leftmargin=*, topsep=0pt]
			\item SELECT TOP(5) WITH TIES Guasti.CategoriaProdotto, Guasti.DataRichiestaIntervento, SUM(Ricambi.CostoAcquisto + Ricambi.CostoInstallazione) AS 'CostoTotale'\\
			FROM (Guasti JOIN Difetti ON Guasti.ComponentCode = Difetti.ComponentCode AND Guasti.CodiceTipoDifetto = Difetti.CodiceTipo) JOIN Ricambi ON Difetti.ComponentCode = Ricambi.ComponentCode AND Difetti.CodiceTipo = Ricambi.CodiceTipoDifetto\\
			WHERE MONTH(Guasti.DataRichiestaIntervento) = MONTH(CURRENT\_TIMESTAMP) AND YEAR(Guasti.DataRichiestaIntervento) = YEAR(CURRENT\_TIMESTAMP)\\
			GROUP BY Guasti.NumeroTelefonoCliente, Guasti.DataRichiestaIntervento, Guasti.CategoriaProdotto ORDER BY 3 DESC;
		\end{itemize}
	\newpage
	\item[\textbf{P7 -}] Le prime cinque zone per numero di guasti avvenuti in questo mese
		\begin{itemize}[leftmargin=*, topsep=0pt]
			\item SELECT Guasti.CategoriaProdotto, Interventi.Zona, COUNT(*) AS '\#'\\
			FROM Guasti JOIN Interventi ON Guasti.NumeroTelefonoCliente = Interventi.NumeroTelefonoCliente AND Guasti.DataRichiestaIntervento = Interventi.DataRichiesta\\
			WHERE Guasti.CodiceTipoDifetto IS NOT NULL AND Guasti.ComponentCode IS NOT NULL AND MONTH(Guasti.DataRichiestaIntervento) = \\MONTH(CURRENT\_TIMESTAMP) AND YEAR(Guasti.DataRichiestaIntervento) = YEAR(CURRENT\_TIMESTAMP)\\
			GROUP BY Interventi.Zona, Guasti.CategoriaProdotto;
		\end{itemize}
	\item[\textbf{P8 -}] I \textit{Time To Failure} rispetto alla data di acquisto di ciascun prodotto coinvolto nei guasti di questo mese
		\begin{itemize}[leftmargin=*, topsep=0pt]
			\item SELECT Guasti.CategoriaProdotto, Prodotti.PNC, Prodotti.SNC, Prodotti.CodiceGaranzia, Prodotti.Modello, DATEDIFF(d, Prodotti.DataAcquisto, Guasti.DataRichiestaIntervento) AS 'TTF (giorni)'\\
			FROM Guasti JOIN Prodotti ON Guasti.PNC = Prodotti.PNC AND Guasti.SNC = Prodotti.SNC\\
			WHERE Guasti.CodiceTipoDifetto IS NOT NULL AND Guasti.ComponentCode IS NOT NULL AND MONTH(Guasti.DataRichiestaIntervento) = \\MONTH(CURRENT\_TIMESTAMP) AND YEAR(Guasti.DataRichiestaIntervento) = YEAR(CURRENT\_TIMESTAMP) AND Prodotti.DataAcquisto IS NOT NULL;
		\end{itemize}
	\item[\textbf{P9 -}] I \textit{Time To Failure} rispetto alla data di installazione di ciascun prodotto coinvolto nei guasti di questo mese
		\begin{itemize}[leftmargin=*, topsep=0pt]
			\item SELECT Guasti.CategoriaProdotto, Prodotti.PNC, Prodotti.SNC, Prodotti.CodiceGaranzia, Prodotti.Modello, DATEDIFF(d, Prodotti.DataInstallazione, Guasti.DataRichiestaIntervento) AS 'TTF (giorni)'\\
			FROM Guasti JOIN Prodotti ON Guasti.PNC = Prodotti.PNC AND Guasti.SNC = Prodotti.SNC\\
			WHERE Guasti.CodiceTipoDifetto IS NOT NULL AND Guasti.ComponentCode IS NOT NULL AND MONTH(Guasti.DataRichiestaIntervento) = \\MONTH(CURRENT\_TIMESTAMP) AND YEAR(Guasti.DataRichiestaIntervento) = YEAR(CURRENT\_TIMESTAMP) AND Prodotti.DataInstallazione IS NOT NULL;
		\end{itemize}
	\newpage
	\item[\textbf{P10 -}] Tempo medio di riparazione di un guasto per tipo di difetto
		\begin{itemize}[leftmargin=*, topsep=0pt]
			\item SELECT Guasti.CategoriaProdotto, Difetti.CodiceTipo, Difetti.NomeTipo, AVG(Interventi.TempoImpiegato) AS 'TempoMedioRiparazione'\\
			FROM (Guasti JOIN Difetti ON Guasti.ComponentCode = Difetti.ComponentCode AND Guasti.CodiceTipoDifetto = Difetti.CodiceTipo) JOIN Interventi ON Guasti.NumeroTelefonoCliente = Interventi.NumeroTelefonoCliente AND Guasti.DataRichiestaIntervento = Interventi.DataRichiesta\\
			WHERE MONTH(Guasti.DataRichiestaIntervento) = MONTH(CURRENT\_TIMESTAMP) AND YEAR(Guasti.DataRichiestaIntervento) = YEAR(CURRENT\_TIMESTAMP) AND Interventi.TempoImpiegato IS NOT NULL\\
			GROUP BY Guasti.CategoriaProdotto, Difetti.CodiceTipo, Difetti.NomeTipo ORDER BY 4 DESC;
		\end{itemize}
	\item[\textbf{V1} -] Conteggio di tutti gli interventi aperti nel mese da ciascun operatore
		\begin{itemize}[leftmargin=*, topsep=0pt]
			\item SELECT Operatori.CF, Operatori.Nome, Operatori.Cognome, COUNT(*) AS '\#'\\
			FROM Interventi JOIN Operatori ON Interventi.CodiceOperatore = Operatori.Codice\\
			WHERE MONTH(Interventi.DataRichiesta) = MONTH(CURRENT\_TIMESTAMP) AND YEAR(Interventi.DataRichiesta) = YEAR(CURRENT\_TIMESTAMP)\\
			GROUP BY Operatori.Codice, Operatori.CF, Operatori.Nome, Operatori.Cognome;
		\end{itemize}
	\item[\textbf{V2 -}] Conteggio di tutti gli interventi chiusi nel mese corrente da ciascun tecnico
		\begin{itemize}[leftmargin=*, topsep=0pt]
			\item SELECT Tecnici.CF, Tecnici.Nome, Tecnici.Cognome, COUNT(*) AS '\#'\\
			FROM Interventi JOIN Tecnici ON Interventi.CodiceTecnico = Tecnici.Codice\\
			WHERE MONTH(Interventi.DataRichiesta) = MONTH(CURRENT\_TIMESTAMP) AND YEAR(Interventi.DataRichiesta) = YEAR(CURRENT\_TIMESTAMP)\\
			GROUP BY Tecnici.Codice, Tecnici.CF, Tecnici.Nome, Tecnici.Cognome;
		\end{itemize}
	\item[\textbf{V3 -}] Tempo medio di riparazione di un guasto per ciascun tecnico
		\begin{itemize}[leftmargin=*, topsep=0pt]
			\item SELECT Tecnici.CF, Tecnici.Nome, Tecnici.Cognome, AVG(Interventi.TempoImpiegato) AS 'TempoMedioImpiegato'\\
			FROM Interventi JOIN Tecnici ON Interventi.CodiceTecnico = Tecnici.Codice\\
			GROUP BY Tecnici.Codice, Tecnici.CF, Tecnici.Nome, Tecnici.Cognome;
		\end{itemize}
	\newpage
	\item[\textbf{V4 -}] Distanza temporale media tra ricezione della chiamata e visita del tecnico per centro assistenza
		\begin{itemize}[leftmargin=*, topsep=0pt]
			\item SELECT Centri\_Assistenza.CodiceNazionale, Centri\_Assistenza.Nazione, Centri\_Assistenza.Sede, Centri\_Assistenza.AreaCompetenza, AVG(CAST(DATEDIFF(d, Interventi.DataRichiesta, Interventi.DataVisita) AS REAL)) AS 'TempoMedioRiparazione'\\
			FROM (Interventi JOIN Operatori ON Interventi.CodiceOperatore = Operatori.Codice) JOIN Centri\_Assistenza ON Operatori.CodiceNazionaleCentro = Centri\_Assistenza.CodiceNazionale AND Operatori.NazioneCentro = Centri\_Assistenza.Nazione\\
			GROUP BY Centri\_Assistenza.CodiceNazionale, Centri\_Assistenza.Nazione, Centri\_Assistenza.Sede, Centri\_Assistenza.AreaCompetenza;
		\end{itemize}
\end{itemize}

\chapter{Progettazione dell'applicazione}

L'applicazione è stata sviluppata attraverso l'utilizzo del linguaggio C\#, fornito attraverso il .Net Framework 4.7.2, avvalendosi in particolar modo del servizio per la realizzazione di \textit{GUI} interattive "\textit{Windows Forms}" per il sistema operativo Windows 10 offerto dal linguaggio stesso. Il \textit{DBMS} sui cui l'applicazione si basa è \textit{Microsoft SQL Server} che gestisce un database contenuto in un server virtuale del servizio cloud "\textit{Microsoft Azure}".\newline
L'applicazione nella sua struttura ricalca la divisione tra gli utenti che utilizzano il database evidenziata in precedenza. Essa può essere suddivisa in cinque schermate.

\section{Schermata di login}

\begin{figure}[H]
	\centering
	\includegraphics[width=\linewidth]{images/loginScreen.png}
	\caption{Immagine della schermata di login}
\end{figure}

La schermata di login è estremamente essenziale e in quanto tale permette un'interazione con l'utente minimale. Qui l'utente può inserire il proprio codice dipendente ed accedere
ad una seconda schermata con i dati di suo interesse, una diversa per ogni categoria di utente che utilizza il database. Oltre alle tre categorie elencate in precedenza - operatori,
tecnici e progettisti - è stata considerata una quarta categoria di utenti, gli amministratori. Gli amministratori sono coloro che eseguono le operazioni che non appartengono ad una
categoria specifica di utenti, perché cercano di ottenere informazioni generali che riguardano i dipendenti, come ad esempio le prestazioni di ognuno di essi.
La loro schermata si preoccuperà perciò di mostrare le operazioni dalla V1 alla V4. Essendo senza un codice dipendente, per accedere alla loro interfaccia
occorre digitare come codice dipendente il simbolo "*".

\section{Schermata degli amministratori}

\begin{figure}[H]
	\centering
	\includegraphics[width=\linewidth]{images/managementScreen.png}
	\caption{Immagine della schermata per gli amministratori}
\end{figure}

Un amministratore può passare in rassegna i dati inerenti alle operazioni di sua competenza scorrendo tra i \textit{tab} che costituiscono la sua schermata.

\section{Schermata dei progettisti}

\begin{figure}[H]
	\centering
	\includegraphics[width=\linewidth]{images/designerScreen.png}
	\caption{Immagine della schermata per i progettisti}
\end{figure}

Un progettista, come un amministratore, può vedere i dati inerenti a ciascuna delle operazioni di sua competenza attraverso le \textit{tab} disponibili nella sua schermata. In particolar
modo, nella \textit{tab} denominata "Cerca per parola chiave" può effettuare una ricerca tra tutti guasti che appartengono alle categorie di sua competenza trattenendo solamente quelli
la cui descrizione fatta dal tecnico contiene una determinata stringa di caratteri.

\section{Schermata degli operatori}

\begin{figure}[H]
	\centering
	\includegraphics[width=\linewidth]{images/operatorScreen.png}
	\caption{Immagine della schermata per gli operatori}
\end{figure}

Questa schermata invece permette una maggior interazione tra l'utente e il sistema, essendo stata progettata per inserire nuovi interventi e relativi guasti. I campi di 
testo iniziali servono per inserire i dati di un cliente e se non è mai stato inserito all'interno del \textit{database} sarà il sistema stesso a deciderlo con un controllo preventivo.
La griglia subito sotto permette di inserire i dati relativi ad un prodotto che ha subito un guasto - e anche in questo caso sarà il sistema in automatico a determinare se è già
stato inserito in passato o meno - nonché sul guasto associato. Utilizzando i dati precedenti assieme a questi ultimi, inoltre, verrà aggiunto un nuovo \textit{record} inerente
all'intervento appena aperto. Subito dopo che l'intervento è stato inserito, in caso di rilevato errore, con le informazioni ancora presenti nel \textit{form}, è possibile rimuovere
dalla base di dati le informazioni inviate tramite il tasto "Cancella". Se invece si è convinti dei dati inseriti, il pulsante "Pulisci" svuota tutti i
campi del \textit{form} e permette di effettuare un nuovo inserimento. Se parte dei dati sono stati rimossi, o alterati, il sistema cancellerà solamente quei dati che riuscirà
a trovare nel \textit{database}, a patto che non siano utili per identificare altri interventi o altri guasti.

\section{Schermata dei tecnici}

\begin{figure}[H]
	\centering
	\includegraphics[width=\linewidth]{images/technicianScreen.png}
	\caption{Immagine della schermata per i tecnici}
\end{figure}

Un tecnico si trova davanti ad una schermata inizialmente con un solo riquadro riempito con una griglia. Questa contiene informazioni sugli interventi che sono ancora aperti e che
possono essere scelti per poter essere chiusi dopo la riparazione. Cliccare sui dati inerenti ad un intervento permetterà la comparsa negli altri due riquadri dei dati inerenti ai
guasti e ai prodotti coinvolti in quello specifico intervento. In particolar modo, compariranno anche i campi vuoti che è responsabilità del tecnico riempire dopo essersi recato a
casa del cliente e dopo aver effettuato la riparazione. Una volta inseriti tutti i dati necessari, verranno inseriti all'interno del database alla pressione del pulsante "Modifica".

\end{document}